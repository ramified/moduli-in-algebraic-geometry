\section{Goal and related concepts}
The personal survey is motivated by the three courses in Bonn: \href{https://johannesschmitt.gitlab.io/moduli_of_curves}{``the moduli space of curves"}, \href{https://www.math.uni-bonn.de/people/mihatsch/21u22\%20WS/moduli/}{``moduli of elliptic curves"} and \href{https://www.math.uni-bonn.de/people/ydutta/v5a4}{``moduli of vector bundles"}. I would also highly recommend the course \href{https://userpage.fu-berlin.de/hoskins/moduli_and_GIT.html}{``Moduli and GIT"} in Freie Universität Berlin and the lecture note \href{https://sites.math.washington.edu/~jarod/}{"Notes on stacks and moduli"} in the University of Washington. This is also partially covered by Dinamo Djounvouna's \href{https://www.math.u-bordeaux.fr/~ybilu/algant/documents/theses/Djounvouna.pdf}{master thesis}.  I want to construct my personal understanding on the moduli, and find out the details I missed in the courses. 

``Some mathematicians are birds, others
are frogs." This document is devoted to those ``birds" from a ``frog" who gets stuck in the mud.
\subsection{Conventions and Notations}
In this survey,  $\Schk$ is denoted as the category of \textbf{locally Noetherian} schemes over $k$, where $k$ is a field or $\mathbb{Z}$ or $\mathbb{Z}\left[\textstyle\frac{1}{6}\right]$.
\subsection{Representable functor}
In this subsection we would follow on \cite[Definition 2.2.1]{huybrechts2010geometry} in full generality, but you can always think 
$$\mathcal{C}=\Schk \qquad \mathcal{C}'=\operatorname{Functor} (\Schk^{op},\Set)$$
to visualize the statement, and you can refer to \cite[6.6.2]{FOAG} to see basic examples.
\begin{defn}[Functor category]
Fix a category $\mathcal{C}$, we define the corresponding functor category $\mathcal{C}'$ as follows:
% https://q.uiver.app/?q=WzAsMixbMCwwLCJcXG1hdGhjYWx7Q31ee29wfSJdLFsxLDAsIlxcU2V0Il0sWzAsMSwiIiwwLHsiY3VydmUiOjF9XSxbMCwxLCIiLDIseyJjdXJ2ZSI6LTF9XSxbMywyLCIiLDIseyJzaG9ydGVuIjp7InNvdXJjZSI6MjAsInRhcmdldCI6MjB9fV1d
$$\Ob(\mathcal{C}'):=\left\{  \text{functors }\mathcal{C}^{op}\longrightarrow \Set \right\} \qquad \Mor(\mathcal{C}'):=\left\{ \text{natural trans }
\begin{tikzcd}
	{\mathcal{C}^{op}} & \Set
	\arrow[""{name=0, anchor=center, inner sep=0}, curve={height=9pt}, from=1-1, to=1-2]
	\arrow[""{name=1, anchor=center, inner sep=0}, curve={height=-9pt}, from=1-1, to=1-2]
	\arrow[shorten <=2pt, shorten >=2pt, Rightarrow, from=1, to=0]
\end{tikzcd}
\right\}$$
In brief, $\mathcal{C}'=\operatorname{Functor} (\mathcal{C}^{op},\Set)$.\footnote{From  \href{https://userpage.fu-berlin.de/hoskins/moduli_and_GIT.html}{here} we can also write $\mathcal{C}'=\Psh(C)$, notice that here it is the presheaf on \textbf{category}, not on a specific scheme $X/k$.}
\end{defn}
\begin{proposition}
We have a canonical functor 
$$\iota: \mathcal{C} \longrightarrow \mathcal{C}' \qquad X \longmapsto h_X:=\Mor_{\mathcal{C}}(-,X)$$ which embeds $\mathcal{C}$ as a full subcategory of $\mathcal{C}'$.\footnote{$h_X$ is called the functor of points of a scheme $X$ from  \href{https://userpage.fu-berlin.de/hoskins/moduli_and_GIT.html}{here}. For example, $h_X(\Spec \mathbb{Q})=X(\mathbb{Q})$ gives us the set of rational points in $X$(under certain conditions). Notice that $h_{-}(-)$ is a bifunctor.}
\end{proposition}
\begin{proof}
Recall that the Yoneda's lemma gives us the isomorphism
$$\Mor_{\mathcal{C}'}(h_X,F)\cong F(X).$$
\end{proof}
\begin{warning}
The functor $\iota: \mathcal{C} \longrightarrow \mathcal{C}'$ may not preserve colimits! For example, for $\mathcal{C}=\Sch_{\mathbb{Z}}$, the colimit of schemes
$$\Spec \mathbb{Z}/p\mathbb{Z} \longrightarrow \Spec \mathbb{Z}/p^2\mathbb{Z} \longrightarrow\Spec \mathbb{Z}/p^2\mathbb{Z} \longrightarrow \cdots$$
is $\Spec \mathbb{Z}_p$ (in $\mathcal{C}$) or $\Spf \mathbb{Z}_p$ (in $\mathcal{C}'$).\footnote{By the universal property we have the canonical map $\Spf \mathbb{Z}_p \longrightarrow \Spec \mathbb{Z}_p$ in $\mathcal{C}'$.}
\end{warning}
\begin{defn}[Representable functor]\label{def:repfctor}
The functor $F\in \Ob(\mathcal{C}')$ is called represented by $X$ if $F\cong h_X$.
\end{defn}
From this proposition, we can always view the object as some functor satisfying some properties. we get three advantages from this point of view:
\begin{itemize}
\item It's easy to see rational points and complex points;
\item We can define scheme canonically, without explicit constructions;
\item We can enlarge our area of research, and think them as the defective schemes. We will see some reasonable functors which is represented not by schemes, but by stacks.
\end{itemize}
\begin{remark}
By \cite[1.3.6]{vistoli2007notes} or \cite[B.2]{alper2021introduction}, any representable functor (after restricted to specific subcategory of $\mathcal{C}=\Schk$) is a sheaf in Zariski/étale/fppf/fpqc topology.
\end{remark}
\subsection{Corepresentable functor}
This is the concept "dual to" the representable functor. To motivate, we begin with the equivalent definition of representable functor:
\begin{defn}[Representatable functor, equivalent definition]
A functor $F\in \Ob(\mathcal{C}')$ is represented by $X \in \Ob(C)$ if $F$ satisfies the following universal properties:
\begin{itemize}
\item There exists a morphism $\eta^{-1}:h_X \longrightarrow F$ in $\Mor_{\mathcal{C}'}(h_X,F)$;
\item For any object $X' \in \Ob(\mathcal{C})$ and morphism $\alpha':h_{X'}\longrightarrow F$ in $\Mor_{\mathcal{C}'}(h_{X'},F)$, there exists an unique morphism $\beta \in \Mor_{\mathcal{C}}(X',X)$ such that $\alpha'=\eta^{-1} \circ h_{\beta}$.
% https://q.uiver.app/?q=WzAsNSxbMCwwXSxbMSwwLCJoX3tYJ30iXSxbMSwxLCJoX3tYfSJdLFsyLDBdLFsyLDEsIkYiXSxbMSwyLCJcXGV4aXN0IFxcLCEgXFwsaF97XFxiZXRhfSIsMix7ImNvbG91ciI6WzM1OCwxMDAsNTBdLCJzdHlsZSI6eyJib2R5Ijp7Im5hbWUiOiJkYXNoZWQifX19LFszNTgsMTAwLDUwLDFdXSxbMiw0LCJcXGFscGhhIiwyXSxbMSw0LCJcXGFscGhhJyJdXQ==
$$\begin{tikzcd}
	{} & {h_{X'}} & {} \\
	& {h_{X}} & F
	\arrow["{\exists \,! \,h_{\beta}}"', color={rgb,255:red,255;green,0;blue,8}, dashed, from=1-2, to=2-2]
	\arrow["\eta^{-1}"', from=2-2, to=2-3]
	\arrow["{\alpha'}", from=1-2, to=2-3]
\end{tikzcd}$$
\end{itemize}
\end{defn}
This definition is equivalent to the Definition \ref{def:repfctor} since \footnote{The middle isomorphism comes from the universal property, while the two equalities come from the Yoneda's lemma.}
$$h_X(X')=\Mor_{\mathcal{C}}(X',X) \cong \Mor_{\mathcal{C}'}(h_{X'},F) =F(X') \text{ for any } X'\in \Ob(\mathcal{C})$$
thus the functor $\eta^{-1}: h_X \longrightarrow F$ is an isomorphism.
\begin{defn}[Corepresentable functor]
A functor $F\in \Ob(\mathcal{C}')$ is corepresented by $X \in \Ob(C)$ if $F$ satisfies the following universal properties:
\begin{itemize}
\item There exists a morphism $\eta:F \longrightarrow h_X$ in $\Mor_{\mathcal{C}'}(F,h_X)$;
\item For any object $X' \in \Ob(\mathcal{C})$ and morphism $\alpha':F\longrightarrow h_{X'}$ in $\Mor_{\mathcal{C}'}(F,h_{X'})$, there exists an unique morphism $\beta \in \Mor_{\mathcal{C}}(X,X')$ such that $\alpha'=h_{\beta} \circ \eta$.
% https://q.uiver.app/?q=WzAsMyxbMCwwLCJGIl0sWzEsMCwiaF9YIl0sWzEsMSwiaF97WCd9Il0sWzAsMSwiXFxhbHBoYSJdLFsxLDIsIlxcZXhpc3QgXFwsISBcXCxoX3tcXGJldGF9IiwwLHsiY29sb3VyIjpbMzU4LDEwMCw1MF0sInN0eWxlIjp7ImJvZHkiOnsibmFtZSI6ImRhc2hlZCJ9fX0sWzM1OCwxMDAsNTAsMV1dLFswLDIsIlxcYWxwaGEnIiwyXV0=
$$\begin{tikzcd}
	F & {h_X} \\
	& {h_{X'}}
	\arrow["\eta", from=1-1, to=1-2]
	\arrow["{\exists \,! \,h_{\beta}}", color={rgb,255:red,255;green,0;blue,8}, dashed, from=1-2, to=2-2]
	\arrow["{\alpha'}"', from=1-1, to=2-2]
\end{tikzcd}$$
\end{itemize}
\end{defn}
\begin{defn}[Universal corepresentable functor]
A functor $F\in \Ob(\mathcal{C}')$ is universal corepresented by $X \in \Ob(C)$ if for any morphism $\gamma :Y\longrightarrow X$, the functor $F \times_{h_X} h_Y$ is corepresented by $Y$.
% https://q.uiver.app/?q=WzAsNCxbMCwxLCJGIl0sWzEsMSwiaF9YIl0sWzEsMCwiaF9ZIl0sWzAsMCwiRiBcXHRpbWVzX3toX1h9IGhfWSJdLFswLDEsIlxcYWxwaGEiXSxbMiwxLCJoX3tcXGdhbW1hfSJdLFszLDJdLFszLDBdXQ==
\[\begin{tikzcd}
	{F \times_{h_X} h_Y} & {h_Y} \\
	F & {h_X}
	\arrow["\alpha", from=2-1, to=2-2]
	\arrow["{h_{\gamma}}", from=1-2, to=2-2]
	\arrow[from=1-1, to=1-2]
	\arrow[from=1-1, to=2-1]
\end{tikzcd}\]
\end{defn}
\begin{proposition}
Suppose a functor $F\in \Ob(\mathcal{C}')$ is corepresented by $X \in \Ob(C)$, then it is represented by $X$ if and only if $\eta:F \longrightarrow h_X$ is a $\mathcal{C}'$-isomorphism.
\end{proposition}
\subsection{Coarse moduli space}
You need to see the definition of the naive moduli problem, extended moduli problem, moduli functor, fine moduli space and corresponding universal family from  \href{https://userpage.fu-berlin.de/hoskins/moduli_and_GIT.html}{here}. I don't want to copy, and it's well-written.

The following example will check if you really understand these concepts.
\begin{eg}[$\mathcal{M}_{0,3}$ is represented by one point $\Spec k$]\label{eg:M03}\

\textbf{Naive moduli problem} $(\mathcal{A},\sim)$:
$$\mathcal{A}:=\left\{(p_1,p_2,p_3)  \;\middle|\; p_i \in \mathbb{P}_k^1(k) \text{ are distinct } \right\}$$
$$(p_1,p_2,p_3) \sim (q_1,q_2,q_3) \Longleftrightarrow \exists \, \gamma \in \PGL_2(k) \text{ s.t. } q_i=\gamma(p_i)$$

\textbf{Extended moduli problem} $(\mathcal{A}_S,\sim_S)$ + pullback: \textcolor{gray}{$(S \in \Ob(\Schk))$}
$$\mathcal{A}_S:=\left\{(X,\pi,\sigma_1,\sigma_2,\sigma_3)  \;\middle|\; \begin{aligned}
&\\[-5mm]
&X \in \Ob(\Schk)\\[-1mm]
&\pi:X \longrightarrow S \text{ proper flat and }\pi^{-1}(p)\cong\mathbb{P}_{\kappa(p)}^1 \\[-1mm]
&\sigma_i:S \longrightarrow X \text{ pairwise disjoint sections of }\pi
\end{aligned}
 \right\}$$
 
 $(X,\pi,\sigma_1,\sigma_2,\sigma_3) \sim_S (X',\pi',\sigma_1',\sigma_2',\sigma_3')$ if there exists an isomorphism $f:X \longrightarrow X'$ such that the following diagrams commute:
   % https://q.uiver.app/?q=WzAsNixbMCwwLCJYIl0sWzIsMCwiWCciXSxbMSwxLCJTIl0sWzMsMCwiWCJdLFs0LDEsIlMiXSxbNSwwLCJYJyJdLFszLDUsImYiXSxbNCw1LCJcXHNpZ21hX2knIiwyXSxbNCwzLCJcXHNpZ21hX2kiXSxbMCwyLCJcXHBpIiwyXSxbMSwyLCJcXHBpJyJdLFswLDEsImYiXV0=
   \[\begin{tikzcd}[column sep=3mm]
   	X && {X'} &[20mm] X && {X'} \\
   	& S &&& S
   	\arrow["f", from=1-4, to=1-6]
   	\arrow["{\sigma_i'}"', from=2-5, to=1-6]
   	\arrow["{\sigma_i}", from=2-5, to=1-4]
   	\arrow["\pi"', from=1-1, to=2-2]
   	\arrow["{\pi'}", from=1-3, to=2-2]
   	\arrow["f", from=1-1, to=1-3]
   \end{tikzcd}\]
   
   For a map $f:T \longrightarrow S$, the pullback $f^*$ is defined by
   $$f^*:\mathcal{A}_S \longrightarrow \mathcal{A}_T \qquad (X,\pi,\sigma_1,\sigma_2,\sigma_3) \longmapsto (X_T,\pi_T,\sigma_{1,T},\sigma_{2,T},\sigma_{3,T})$$
   where $X_T$, $\pi_T$ are defined as fiber product, and $\sigma_{i,T}$ are defined by the universal property of fiber product, as follows:
   % https://q.uiver.app/?q=WzAsOSxbMCwxLCJYX1QiXSxbMSwxLCJYIl0sWzAsMiwiVCJdLFsxLDIsIlMiXSxbMywxLCJYX1QiXSxbMywyLCJUIl0sWzQsMiwiUyJdLFs0LDEsIlgiXSxbMiwwLCJUIl0sWzAsMiwiXFxwaV9UIiwyXSxbMiwzLCJmIl0sWzEsMywiXFxwaSJdLFs0LDUsIlxccGlfVCIsMl0sWzUsNiwiZiJdLFs3LDYsIlxccGkiLDJdLFs0LDddLFs4LDcsIlxcc2lnbWFfaSBcXGNpcmMgZiIsMCx7ImN1cnZlIjotMn1dLFs4LDQsIlxcZXhpc3RzIFxcLCFcXCwgXFxzaWdtYV97aSxUfSAiLDAseyJsYWJlbF9wb3NpdGlvbiI6OTAsImNvbG91ciI6WzAsMTAwLDUwXSwic3R5bGUiOnsiYm9keSI6eyJuYW1lIjoiZGFzaGVkIn19fSxbMCwxMDAsNTAsMV1dLFs4LDUsIlxcSWRfVCIsMix7ImN1cnZlIjoyfV0sWzYsNywiXFxzaWdtYV97aX0iLDIseyJjdXJ2ZSI6MSwic3R5bGUiOnsiYm9keSI6eyJuYW1lIjoiZGFzaGVkIn19fV0sWzAsMV0sWzAsMywiIiwxLHsic3R5bGUiOnsibmFtZSI6ImNvcm5lciJ9fV0sWzQsNiwiIiwxLHsic3R5bGUiOnsibmFtZSI6ImNvcm5lciJ9fV1d
   \[\begin{tikzcd}
   	&&[15mm] T&[-6mm] \\[-6mm]
   	{X_T} & X && {X_T} & X \\
   	T & S && T & S
   	\arrow["{\pi_T}"', from=2-1, to=3-1]
   	\arrow["f", from=3-1, to=3-2]
   	\arrow["\pi", from=2-2, to=3-2]
   	\arrow["{\pi_T}"', from=2-4, to=3-4]
   	\arrow["f", from=3-4, to=3-5]
   	\arrow["\pi"', from=2-5, to=3-5]
   	\arrow[from=2-4, to=2-5]
   	\arrow["{\sigma_i \circ f}", curve={height=-12pt}, from=1-3, to=2-5]
   	\arrow["{\exists \,!\, \sigma_{i,T} }"{pos=1.12}, color=red, dashed, from=1-3, to=2-4]
   	\arrow["{\Id_T}"', curve={height=12pt}, from=1-3, to=3-4]
   	\arrow["{\sigma_{i}}"', curve={height=6pt}, dashed, from=3-5, to=2-5]
   	\arrow[from=2-1, to=2-2]
   	\arrow["\ulcorner"{anchor=center, pos=0.125}, draw=none, from=2-1, to=3-2]
   	\arrow["\ulcorner"{anchor=center, pos=0.125}, draw=none, from=2-4, to=3-5]
   \end{tikzcd}\]
 \begin{adjustwidth}{10mm}{0em}
 \begin{remarkstar}
 We can always view the point $p \in S$ as an affine scheme of its residue field. So the two conditions below are equivalent:
 \begin{equation*}
 \begin{aligned}
  &\pi^{-1}(p)\cong\mathbb{P}_{\kappa(p)}^1& & \text{ for any } p \in S  \\ 
  &\pi^{-1}(\Spec L)\cong \mathbb{P}_{L}^1& & \text{ for any } \Spec L \hookrightarrow S  \\ 
 \end{aligned}
 \end{equation*}
 \end{remarkstar}
 \begin{remarkstar}
 In some textbooks, they require $\pi:X \longrightarrow S$ to be a smooth, proper, surjective, locally finitely presented (l.f.p.)
 morphism of relative dimension $\leqslant 1$ with geometric fibres isomorphic to $\mathbb{P}^1$. These extra conditions are automatically satisfied, since
 % https://q.uiver.app/?q=WzAsNixbMCwwLCJcXGJ1bGxldCJdLFsxLDAsIlxcdGV4dHtzdXJqZWN0aXZlfSJdLFsxLDEsIlxcYnVsbGV0Il0sWzAsMSwiXFxidWxsZXQiXSxbMSwyLCJcXGJ1bGxldCJdLFswLDIsIlxcYnVsbGV0Il0sWzAsMV0sWzAsMl0sWzMsMl0sWzMsNF0sWzUsNF1d
 \[\begin{tikzcd}[/tikz/column 1/.append style={anchor=base east},/tikz/column 2/.append style={anchor=base west}, column sep=3cm, row sep=-2mm,start anchor=east,end anchor=west]
 	\pi^{-1}(p)\cong\mathbb{P}_{\kappa(p)}^1 & {\text{surjective}} \\
 	{\text{proper}} & {\text{relative dimension $\leqslant 1$}} \\
 	{\text{locally Noetherian}} & {\text{locally finitely presented}}
 	\arrow[from=1-1, to=1-2]
 	\arrow[from=1-1, to=2-2]
 	\arrow[from=2-1, to=2-2]
 	\arrow[from=2-1, to=3-2]
 	\arrow[from=3-1, to=3-2]
 \end{tikzcd}\]
 and by \cite[25.2.2]{FOAG}, flat + fiberwise $\mathbb{P}^1$ + l.f.p. $\Rightarrow$ smooth. 
 
 \end{remarkstar}
\begin{problem}\hspace{1cm}

Can we find $(X,\pi,\sigma_1,\sigma_2,\sigma_3)$ satisfying all conditions in $\mathcal{A}_S$ except $\pi$ is separated? 

Can we find $(X,\pi,\sigma_1,\sigma_2,\sigma_3)$ satisfying all conditions in $\mathcal{A}_S$ except $\pi$ is flat?
\end{problem}
\begin{answer}
Still unknown. From Dr. Johannes Anschütz, the map
$$\mathbb{P}_k^1 \longrightarrow \Spec k \longrightarrow \Spec k[t]/t^2$$
is proper but not flat(sadly we can't find any section). If the base scheme $S$ is reduced, then maybe the flatness can be checked fiberwise?

Another obstruction for the flatness comes from \cite[26.2.11]{FOAG} or \href{https://math.stackexchange.com/questions/1704423/smooth-fibration-on-a-smooth-curve}{stackexchange}.
\end{answer}
\begin{easyex}\hspace{1cm}

Verify that $(\mathcal{A}_S,\sim_S)$ and pullback satisfy (i)-(iv) in \cite[Definition 2.10]{moduliGIT}.
\end{easyex}



 \end{adjustwidth}



\textbf{Moduli functor} $\mathcal{M}_{0,3}$:
The moduli functor $\mathcal{M}_{0,3}: \Schk^{op}\longrightarrow \Set$ is defined as
$$\mathcal{M}_{0,3}(S)=\mathcal{A}_S/\sim_S \qquad \mathcal{M}_{0,3}(f:T\longrightarrow S)=f^*: \mathcal{A}_S/\sim_S \longrightarrow \mathcal{A}_T/\sim_T.$$
\vspace{-1.1cm}
\begin{adjustwidth}{10mm}{0em}

\begin{problem}\hspace{1cm}

Why is the moduli functor $\mathcal{M}_{0,3}$ represented by $\Spec k$?
\end{problem}
\begin{answer}\hspace{1cm}

Yes, but it's pretty hard to prove it. The proof of \cite[Proposition 4.1]{modulicurve} uses \cite[Proposition 4.2]{modulicurve} whose proof is quite technical.

Actually we construct the natural functor $\eta^{-1}:h_{\Spec k} \longrightarrow \mathcal{M}_{0,3}$ by
$$\eta^{-1}_S:h_{\Spec k}(S) \longrightarrow \mathcal{M}_{0,3}(S) \qquad [S \rightarrow \Spec k] \longmapsto (\mathbb{P}_S^1=\mathbb{P}_k^1 \times_k S, pr_1,0,1,\infty)$$
and then verify that $\eta^{-1}_S$ is an isomorphism.
\end{answer}
\end{adjustwidth}
\textbf{Universal family}:
By \cite[Definition 2.16]{moduliGIT}, the universal family is $(\mathbb{P}_k^1, \pi,0,1,\infty)$.

\end{eg}

\begin{remark}
Whenever the fine moduli is mentioned, the following slogan is also mentioned:
  \begin{center}
  	\noindent\fbox{\parbox{.7\textwidth}{%
  	The presence of nontrivial automorphisms \\
  	often prevents the existence of a fine moduli space.
  	}}
  \end{center} 
  Some detailed explanations can be found in [\href{http://ncatlab.org/nlab/show/moduli+space#because}{ncatlab}], \cite[Remark 2.11]{modulicurve} and \cite[2.A]{harris2006moduli}. Usually the slogan is achieved as a proof in this way:
   \begin{center}
   	\noindent\fbox{\parbox{.7\textwidth}{%
   	The presence of nontrivial automorphisms \\
   	may help us to construct a nontrivial locally-trivial family,\\
   	which prevents the existence of a fine moduli space.
   	}}
   \end{center}  
   It is, however, generally not easy to construct such a family. According to the comment \href{https://math.stackexchange.com/questions/609131/killing-the-automorphisms-to-make-a-functor-representable}{here}, the problem mainly comes from carefully handling the quotient in scheme category, and tricks to show the ``non-trivial'' keyword. Some discussions about construction have taken place in stackexchange(\href{https://math.stackexchange.com/questions/1257517/why-the-existence-of-automorphism-of-varieties-makes-a-functor-not-being-a-fine}{1257517}, \href{https://math.stackexchange.com/questions/2860087/any-representable-moduli-problem-of-elliptic-curves-is-rigid}{rigid}) and mathoverflow(\href{https://mathoverflow.net/questions/15884/how-to-use-automorphisms-to-produce-isotrivial-non-trivial-families}{15884}, \href{https://mathoverflow.net/questions/8812/why-do-automorphism-groups-of-algebraic-varieties-have-natural-algebraic-group-s}{8812}).
   
   In concreted cases people can also achieve their goal in a different way. They may produce two non-isomorphic objects which become isomorphic after the field extension, or use descent theory to compute automorphism group(See \href{https://www.math.uni-bonn.de/people/mihatsch/21u22\%20WS/moduli/13\%20-\%20Examples\%20of\%20Descent.pdf}{Lec 13}, p5, Application).
   %Or show no non-trivial automorphism by the uniqueness of Weierstrass function
   %Of course people can define automorphism groups which is compatible to extended moduli problem, but that is not useful. The pullback is not uniquely determined.
\end{remark}
\begin{problem}
The following problems may be not well-stated. 
\begin{itemize}
\item Construct a scheme $X$ with $\mathbb{Z}$-free action on $X$, such that the quotient space is still a scheme. (It should be used to construct $E \times X/\mathbb{Z} \longrightarrow X/\mathbb{Z}$)
\item Summerize methods to prove if a family is trivial or not.
\item Show that any elliptic curve over an intergral scheme $S$ has finite automorphisms. Or give a counterexample. (Maybe solved in \cite[Prop 5.3]{deligne1975courbes})
\end{itemize}
\end{problem}
coarse moduli space and some related concepts. 
\subsection{Bigger category}
I would follow \cite{alper2021introduction}. The \href{https://maths-people.anu.edu.au/~alperj/papers/stacks-guide.pdf}{guide} by Jarod Alper may be also useful.

We need to define Grothendieck Topology and site, discussing how this generalized the language of scheme. 

We need to define stack, algebraic stack, Deligne-Munford stack and some related concepts. 
\begin{remark}
When is a coarse moduli space also a fine moduli space? Take a look on \href{https://mathoverflow.net/questions/10532/when-is-a-coarse-moduli-space-also-a-fine-moduli-space}{mathoverflow} or \href{https://math.stackexchange.com/questions/323190/coarse-moduli-space-with-no-autmorphisms-is-also-a-fine-moduli-space}{stackexchange}.

When is an algebraic space a scheme? Take a look on \href{https://mathoverflow.net/questions/4573/when-is-an-algebraic-space-a-scheme}{mathoverflow}.
\end{remark}
\subsection{Goal}
???
\subsection{Quotient}
We will frequently use the quotient. Here is the list of quotients we've already seen:
\begin{itemize}
\item (We could begin at topology: quotient by a subset) linear space-> Abelian category,group,ring(Notice that for this item, we don't take quotient by a group, so sometimes it looks easier)
\item topology(This gives us an categorical quotient!)
\item manifold
\item scheme(Categorical quotient; GIT quotient)
\end{itemize}
See \cite{moduliGIT} for the following concepts:
\begin{itemize}
\item categorical quotients, good quotients, geometric quotients, and GIT quotients.
\item reductive, linearly reductive, and geometrically reductive.(see \cite[Theorem 4.16]{moduliGIT})
\item stable, semistable, unstable, and polystable.
\item Hilbert-Mumford Criterion and Fundamental Theorem in GIT.
\end{itemize}
See \cite{edixhoven2012abelian} for the following concepts:
\begin{itemize}
\item universal categorical quotients, universal geometric quotients.(4.5(iv), 4.12)
\item fppf quotient.(4.28)
\end{itemize}
% % https://q.uiver.app/?q=WzAsOSxbMCw0LCJcXHN1YnN0YWNre1xcdGV4dHtnZW9tZXRyaWMgcXVvdGllbnR9fSJdLFsxLDQsIlxcc3Vic3RhY2t7XFx0ZXh0e2dvb2QgcXVvdGllbnR9fSJdLFsyLDQsIlxcc3Vic3RhY2t7XFx0ZXh0e2NhdGVnb3JpY2FsIHF1b3RpZW50fX0iXSxbMSwzLCJcXHN1YnN0YWNre1xcdGV4dHtyZWR1Y3RpdmUgR0lUIHF1b3RpZW50fX0iXSxbMiwyLCJcXHN1YnN0YWNre1xcdGV4dHt1bml2ZXJzYWx9XFxcXCBcXHRleHR7Y2F0ZWdvcnVjYWwgcXVvdGllbnR9fSJdLFswLDIsIlxcc3Vic3RhY2t7XFx0ZXh0e3VuaXZlcnNhbH1cXFxcIFxcdGV4dHtnZW9tZXRyaWMgcXVvdGllbnR9fSJdLFswLDEsIlxcc3Vic3RhY2t7XFx0ZXh0e2ZwcGYgcXVvdGllbnR9fSJdLFsxLDEsIlxcc3Vic3RhY2t7XFx0ZXh0e2ZwcWMgcXVvdGllbnR9fSJdLFswLDAsIlxcc3Vic3RhY2t7XFx0ZXh0e8OpdGFsZSBxdW90aWVudH19Il0sWzAsMSwiIiwwLHsibGV2ZWwiOjJ9XSxbMSwyLCIiLDAseyJsZXZlbCI6Mn1dLFszLDEsIlxcc3Vic3RhY2t7XFx0ZXh0e1RobSA0LjMwfX0iLDAseyJsZXZlbCI6Mn1dLFszLDAsIlxcc3Vic3RhY2t7XFx0ZXh0e3N0YWJsZX19IiwwLHsic3R5bGUiOnsiYm9keSI6eyJuYW1lIjoiZGFzaGVkIn19fV0sWzUsNCwiIiwwLHsibGV2ZWwiOjJ9XSxbNSwwLCIiLDAseyJsZXZlbCI6Mn1dLFs0LDIsIiIsMSx7ImxldmVsIjoyfV0sWzYsNSwiIiwwLHsibGV2ZWwiOjJ9XSxbMCw2LCJcXHN1YnN0YWNre1xcdGV4dHthc3N1bXB0aW9uIGlufSBcXFxcIFxcdGV4dHtcXGNpdGVbKDQuMzUpKGlpKV17ZWRpeGhvdmVuMjAxMmFiZWxpYW59fX0iLDAseyJvZmZzZXQiOi01LCJjdXJ2ZSI6LTUsInN0eWxlIjp7ImJvZHkiOnsibmFtZSI6ImRhc2hlZCJ9fX1dLFs2LDcsIiIsMCx7ImxldmVsIjoyfV0sWzgsNiwiIiwwLHsibGV2ZWwiOjJ9XV0=
\[\begin{tikzcd}
	{\substack{\text{étale quotient}}} \\
	{\substack{\text{fppf quotient}}} & {\substack{\text{fpqc quotient}}} \\
	{\substack{\text{universal}\\ \text{geometric quotient}}} && {\substack{\text{universal}\\ \text{categorucal quotient}}} \\
	& {\substack{\text{reductive GIT quotient}}} \\
	{\substack{\text{geometric quotient}}} & {\substack{\text{good quotient}}} & {\substack{\text{categorical quotient}}}
	\arrow[Rightarrow, from=5-1, to=5-2]
	\arrow[Rightarrow, from=5-2, to=5-3]
	\arrow["{\substack{\text{Thm 4.30}}}", Rightarrow, from=4-2, to=5-2]
	\arrow["{\substack{\text{stable}}}", dashed, from=4-2, to=5-1]
	\arrow[Rightarrow, from=3-1, to=3-3]
	\arrow[Rightarrow, from=3-1, to=5-1]
	\arrow[Rightarrow, from=3-3, to=5-3]
	\arrow[Rightarrow, from=2-1, to=3-1]
	\arrow["{\substack{\text{assumption in} \\ \text{\cite[(4.35)(ii)]{edixhoven2012abelian}}}}", shift left=5, curve={height=-30pt}, dashed, from=5-1, to=2-1]
	\arrow[Rightarrow, from=2-1, to=2-2]
	\arrow[Rightarrow, from=1-1, to=2-1]
\end{tikzcd}\]
